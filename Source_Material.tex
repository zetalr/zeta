\documentclass[UTF8]{ctexart}
\title{素材}
\author{Z.L.}
\begin{document}
\maketitle\section{创新与发展}
\subsection{名人名言}
1.创新是科学房屋的生命力——阿西莫夫\par
2.距离已经消失,要么创新,要么死亡——托马斯彼得斯\par
3.发展才是硬道理——邓小平\par
4.没有创新,就不可能有合理的,尤其是有效的管理。——阿法纳西耶夫\par
5.要创新需要一定的灵感,这灵感不是天生的,而是来自长期的积累与全身心的投入。没有积累就不会有创新。——王亚宁\par
6.想出新办法的人在他的办法没有成功以前,人家总说他是异想天开。——马克·吐温\par
7.苟日新,日新,又日新。——汉·《礼记·大学》\par
8.不慕古,不留今,与时变,与俗化。——先秦·《管子·正世》\par
9.礼义法度者,应时而变者也。——先秦·《庄子·天运》\par
10.在自然科学中,创方立法,研究某种重要的实验条件,往往要比发现个别事实更有价值——巴甫洛夫\par
11.致富的秘诀,在于"大胆创新、眼光独到"八个大字——陈玉书\par
12.没有创新,就不可能有合理的,尤其是有效的管理——阿法纳西耶夫\par
13.在科学上,每一条道路都应该走一走。发现一条走不通的道路,就是对于科学的一大贡献。——爱因斯坦\par
14.科学研究工作,尤其富于创造性的意义,尤其是要依靠自力更生。当然,自力更生并不等于封锁自己。——李四光\par
15.创新就是在生活中发现了古人没有发现的东西——李可染\par
16.非经自己努力所得的创新,就不是真正的创新——松下幸之助\par
\subsection{相关事例}
1.两会新闻中心首次实现5G信号全覆盖,引发外媒广泛热议。随着5G的兴起,我们再一次看到技术变革带来的重大改变和机遇。5G时代,不仅意味着无处不在的"超级网速",还将为人工智能、无人驾驶、智能家居、虚拟现实等新科技深度融入生活打开大门。\par
2."在我们的国家修建一座医院需要一到两年,而在中国只需要十天!"这是一位外国人对于中国速度的感叹。今年年初,我们只用了十天时间就在武汉修建了火神山、雷神山两座医院,这项成就至今仍然让人感觉难以置信。\par
这个项目时间紧任务重,对病房标准要求也特别高,但是在这样困难重重的条件下,一方有难八方支援,我们越困难就越团结,一座又一座设施齐全的医院便拔地而起,这不仅是我们成功扭转被动局面赢得最终全面胜利的关键节点,而且我们还用行动让世界看到了中国基建的奇迹!\par
3.创新要勇于否定权威(普朗克和爱因斯坦)\par
1900年,著名教授普朗克和儿子在自己的花园里散步。他神情沮丧,很遗憾地对儿子说:"孩子,十分遗憾,今天有个发现。它和牛顿的发现同样重要。"他提出了量子力学假设及普朗克公式。他沮丧这一发现破坏了他一直崇拜并虔诚地信奉为权威的牛顿的完美理论。他终于宣布取消自己的假设。人类本应因权威而受益,却不料竟因权威而受害,由此使物理学理论停滞了几十年。\par
25岁的爱因斯坦敢于冲破权威圣圈,大胆突进,赞赏普朗克假设并向纵深引申,提出了光量子理论,奠定了量子力学的基础。随后又锐意破坏了牛顿的绝对时间和空间的理论,创立了震惊世界的相对论,一举成名,成了一个更伟大的新权威。\par
4.鲁班发明锯\par
相传有一年,鲁班接受了一项建筑一座巨大宫殿的任务。这座宫殿需要很多木料,他和徒弟们只好上山用斧头砍木,当时还没有锯子,效率非常低。一次上山的时候,由于他不小心,无意中抓了一把山上长的一种野草,却一下子将手划破了。鲁班很奇怪,一根小草为什么这样锋利?于是他摘下了一片叶子来细心观察,发现叶子两边长着许多小细齿,用手轻轻一摸,这些小细齿非常锋利。后来,鲁班又看到一条大蝗虫在一株草上啃吃叶子,两颗大板牙非常锋利,一开一合,很快就吃下一大片。这同样引起了鲁班的好奇心,他抓住一只蝗虫,仔细观察蝗虫牙齿的结构,发现蝗虫的两颗大板牙上同样排列着许多小细齿,蝗虫正是靠这些小细齿来咬断草叶的。这两件事给了鲁班很大启发。于是他就用大毛竹做成一条带有许多小锯齿的竹片,然后到小树上去做试验,结果果然不错,几下子就把树杆划出一道深沟,鲁班非常高兴。但是由于竹片比较软,强度比较差,不能长久使用,拉了一会儿,小锯齿就有的断了,有的变钝了,需要更换竹片。鲁班想到了铁片,便请铁匠帮助制作带有小锯齿的铁片。鲁班和徒弟各拉一端,在一棵树上拉了起来,只见他俩一来一往,不一会儿就把树锯断了,又快又省力,锯就这样发明了。\par
5.公元前359年和公元前350年,在商鞅主持下秦国两次公布了新法。秦国经过商鞅变法,面貌焕然一新。秦国从落后国家,一跃而为"兵革大强,诸侯畏惧"的强国,出现了"家给人足,民勇于公战,怯于私斗,乡邑大治"的局面。\par
正是由于它的作用,秦朝的历史才变得如此辉煌。\par
6.1877年,爱迪生发明了留声机,这使他名扬四海。1878年,爱迪生开始了对白炽灯的研究,在十几个月中经过多次失败后,于1879年10月21日成功地点亮了白炽炭丝灯,稳定地点亮了两整天。1882年,在纽约珍珠街创办世界第二座公用火电厂,建立起纽约市区电灯照明系统,成为现代电力系统的雏形。电照明的实现,不仅大大改善了人们生产劳动的条件,也预示着日常生活电气化时代即将到来。1883年,爱迪生在试验真空灯泡时,意外地发现冷、热电极间有电流通过。这种现象后来称为爱迪生效应,成为电子管和电子工业的基础。\par
7.2020年7月31日,中国自主建设、独立运行的全球卫星导航系统全面建成;作为我国迄今为止规模最大、覆盖范围最广、服务性能最高、与百姓生活关联最紧密的巨型复杂航天系统,北斗系统由卫星、火箭、发射场、测控、运控、星间链路、应用验证七大系统组成。从北斗一号工程立项开始,26年的发展历程,才有了今天的成就。\par
8."玲龙一号"全球首个陆上商用小堆目前正在建设,作为我国自主研发并具有自主知识产权的多功能模块化小型堆技术方面走在了世界前列。按照国际原子能机构的定义,发电功率在30万千瓦以下的反应堆为小型堆,"玲龙一号"发电功率为12.5万千瓦,预计2026年建成发电,年发电量可达10亿千瓦时,能满足52.6万户家庭的生活用电需求。牛顿是世界上最伟大的科学家之一,他对科学的贡献是史无前例的。他的一生有许多重大的发现:力学三定律、万有引力、冷却定律以及微积分等。然而到了晚年,他的研究陷入了亚里士多德和柏拉图学说的范畴而不能自拔。他花了十年的时间研究上帝的存在,结果自然毫无所得。由此看来,即使一个伟大的学者,一旦落入陈旧的范畴,也谈不上有丝毫的成就。\par
9.创新是海尔发展的不竭动力。海尔集团始终坚持以技术创新作为发展的手段和依托,在十几年的发展过程中,从引进技术到整合国内外资源、自主创新,坚持"技术创新课题来自于市场难题"和"设计创造高质量、高附加值"的研发理念,通过技术创新使集团在中国市场和国际市场上取得长期的成功,营业额年平均增长率达到78,持续保持在家电与其他相关领域的领先地位。\par
10.我国著名画家齐白石,曾荣获世界和平奖。然而,面对已经取得的成功,他并不满足,而是不断汲取历代画家的长处,不断改进自己作品的风格。他60岁以后的画,明显不同于60岁以前。70岁以后,他的画风又变了一次。80岁以后,他的画风再度变化。齐白石一生,曾五易画风。正因为白石老人在成功后,能仍然马不停蹄地改变、创新,所以他晚年的作品比早期的作品更完美成熟,也形成了自己独特的流派与风格。\par
\newpage
\section{诚信}
\subsection{名人名言}
1.人而无信,不知其可也。——孔子\par
2.没言必诚信,行必忠正。——孔子\par
3.有诚信,何来尊严。——西塞罗\par
4.一言之美,贵于千金。——葛洪\par
5.精诚所至,金石为开。——王充\par
6.如果要别人诚信,首先要自己诚信。——莎士比亚\par
7.伟大人格的素质,重要的是一个诚字。——鲁迅\par
8.坦白是诚实和勇敢的产物。——马克·吐温\par
9.一丝一毫关乎节操,一件小事、一次不经意的失信,可能会毁了我们一生的名誉。——林达生\par
10.金钱比起一分纯洁的良心来,有算得了什么呢?——哈代\par
11.真诚与朴实是天才的宝贵品质——斯坦尼斯拉夫斯基\par
12.当信用消失的时候,肉体就没有生命——大仲马\par
13.真话说一半常是弥天大谎——富兰克林\par
14.真诚才是人生最高的美德——乔叟\par
15.诚实比一切智谋更好,而且它是智谋的基本条件——康德\par
16.诚信为人之本。——鲁迅\par
17.民无信不立。——孔子\par
18.诚者,天之道也;思诚者,人之道也。——孟子\par
19.诚实是力量的一种象征,它显示着一个人的高度自重和内心的安全感与尊严感——艾琳•卡瑟\par
20.不须犯一口说,不须着一意念,只凭真真诚诚行将去,久则自有不言之信,默成之孚。——吕埔\par
21.我宁愿以诚挚获得百名敌人的攻击,也不愿以伪善获得十个朋友的赞扬。——裴多菲\par
\subsection{相关事例}
1.2007年4月陈美丽丈夫因救火不幸身亡后,贴出通告帮亡夫还债。陈美丽坚持诚信为本,无论是手持借据的,还是没有借条的,她都一一偿还。她宁愿自己过苦日子,宁愿自己打工赚钱,也要还清亡夫所欠。她那种不仅自己讲诚信,也信他人的高贵品质,那种人死债不烂、欠债就要还的朴素思想,感动了亿万人。网友称她为当下"中国最诚实守信的村妇"。\par
2.百事可乐的总裁卡尔威勒欧普到科罗拉多大学演讲的时候,有一个名叫杰夫的商人透过演讲会的主办者约卡尔见面谈一谈。卡尔答应了,但只能在演讲后而且只有15分钟的时间。杰夫就在大学礼堂的外面坐等。\par
卡尔兴致勃勃地为大学生们演讲,讲他的创业史,讲商业成功务必遵守的原则,不知不觉中时间已超过了与杰夫约定的见面时间,显然他已忘记了与别人的约定。正当卡尔继续兴致很高地演讲时,他发现一个人从礼堂外推门,径直朝讲台上走来。那人一向走到他的面前,一言不发放下一张名片后转身离去。卡尔拿起名片一看,背面写着:"您和杰夫荷伊在下午两点半有约在先。"卡尔猛然省悟。有一位名不见经传的商人在等他。卡尔没有犹豫,他对大学生们说:"谢谢大家来听我的讲演,本来我还想和大家继续探讨一些问题的,但我有一个约会,而且此刻已经迟到了。迟到已经是对别人的不礼貌,我不能失约,所以请大家原谅,并祝大家好运。"\par
在雷鸣般的掌声中,卡尔快步走出礼堂,他在外面找到了正在等他的杰夫,向他致了歉意后,便又滔滔不绝地告诉了杰夫他所想要明白的一切。结果,原先定好的15分钟时间他们一向交谈了30分钟之后,杰夫成了一名成功的商人,他把这一段经历告诉了他的朋友。他的朋友们都对百事可乐产生了信任并决定经销和宣传百事可乐。\par
3.王一硕,河南省长垣县人。他出生在一个贫穷的农民家庭,,家里唯一的经济来源就是十几亩地的收入。2000年接到大学录取通知书时,面对每年6000元的学费,为了不让父母为难,也为了不使正在上学的两个妹妹失学,他决定放弃上学,去西安打工。当学院得知他是因交不起学费才没来报到时,迅速为他争取了国家助学贷款。他带着万分的感激迈入大学校门。\par
在学校里,领导、老师们一直十分关心王一硕,不仅为他安排勤工助学岗位、为他捐衣捐钱,,更激励他自强不息、不畏艰苦。无论在教室、图书馆,还是实验室,王一硕抓紧一切时间刻苦学习。知恩图报是做人的基本道德,在国家急需人才的时候,自己不能无动于衷。王一硕决定放弃到北京工作的机会,到西部去,为祖国的西部大开发贡献自己的微薄之力。经过层层选拔,2003年8月,王一硕被分配到陕西省麟游县做志愿者。\par
服务结束后,为了实现自己还贷的承诺,王一硕回到郑州,一边复习考研一边打工。随着他的收入不断增加,归还国家助学贷款的愿望也日渐强烈。虽然还有10个多月贷款才到期,但是他决定提前还贷,并向学校正式提出申请。当他将辛苦劳动积攒的26770元贷款交到发费银行负责人手里时,全场响起热烈的掌声。王一硕说:"我现在有能力还清贷款了,我决不会赖账不还。"他以自己的实际行动践行了自己的诺言。\par
4.李国楚任村党支部书记期间积极发展村集体经济,因企业亏损主动辞职并承担企业所欠债务,在荒无人烟、海拔1400多米的深山里种植天麻,历经14年艰辛,面对重重坎坷挫折始终不改初衷,用自己的血汗钱还清了一笔本不该由他承担的集体债,用一诺千金的铿锵行动兑现了他对村民的庄严承诺,被誉为峡江第一"诚信楷模"。\par
5.2010年2月9日,在京、津做建筑工程的孙水林驾车带着妻子、三个儿女和26万元现金从天津出发,准备赶回老家过年,同时给先期回汉的农民工们发放工钱。次日凌晨行至南兰高速开封县陇海铁路桥路段时,由于路面结冰,发生重大车祸,20多辆车连环追尾,孙水林一家五口遇难。孙东林为了完成哥哥的遗愿,顾不上安慰年迈的父母,在腊月二十九将工钱送到60多名农民工手中。由于哥哥的账单多已找不到,孙东林让农民工们凭着良心报领工钱,还贴上了自己的6.6万元和母亲的1万元。孙水林、孙东林兄弟20年信守承诺,被人们誉为"信义兄弟"\par
6.华盛顿用小斧头砍倒了他父亲的一颗樱桃树。父亲见心爱的树被砍,非常气愤,扬言要给那个砍树的一顿教训。而华盛顿在盛怒的父亲面前毫不避地承认了自己的错误。父亲被感动了,称华盛顿的诚实比所有樱桃树都宝贵得多。同样是美国总统尼克松因在"水门事件"中撒谎败露而被迫引咎辞职;克林顿也因为不光彩的绯闻案中撒谎而险遭弹劾。一个因诚实而受到爱戴和尊敬,两位因撒谎而在政史上留下污点。\par
7.一个星期天,宋庆龄一家用过早餐后,就准备到父亲宋耀如的一位朋友家做客,小庆龄听了,高兴地一蹦三尺高。她最喜欢到那位叔叔家了,叔叔家养了鸽子可漂亮了,那位叔叔还说要送她一只呢!小庆龄正准备和爸爸出门时,她突然想起要教好朋友小珍学做花篮。小庆龄便把此事告诉了爸爸,爸爸和她姐姐都让庆龄明天教小珍做花篮,但庆龄说什么也要今天教,父亲听了心里很高兴,还对其他孩子说要向庆龄学习。父亲到了朋友家,把事告诉了他的朋友,那位叔叔也很高兴,还让父亲带回两只鸽子,算是对的她的奖励。\par
8.古代,有个叫孟信的人,一次被罢免了官职以后,家里很穷,甚至连吃的东西都没有了。\par
一天,家里人趁孟信外出把家里仅有的一头病牛卖了,来换粮食。孟信回家后发现病牛被卖了,就把家里人打了一顿,还去把病牛要了回来,他对买主说这是病牛,没什么用处了,这样的病牛不能卖给你。\par
9.立一根柱子在城门前,让人搬,说谁搬了给10两金子,大家都不信,与是提高价格到20两还是没人,几次后价格到了50两,终于有人一式,果然,商鞅马上就给了他50金子。\par
大家都知道了商鞅讲诚信,变法也就方便多了。\par
\newpage
\section{奋斗与追梦}
\subsection{名人名言}
1.天下绝无不热烈勇敢地追求成功,而能取得成功的人。——拿破仑\par
2.无论做什么事情,只要肯发奋奋斗,是没有不成功的。——牛顿\par
3.所有坚韧不拔的发奋迟早会取得报酬的。——安格尔\par
4.奋斗是万物之父。——陶行知\par
5.燧石受到的敲打越厉害,发出的光就越灿烂。——马克思\par
6.老骥伏枥,志在千里;烈士暮年,壮心不已。——曹操\par
7.一块砖没有什么用,一堆砖也没有什么用,如果你心中没有一个造房子的梦想,拥有天下所有的砖头也只是占据了一堆废物;但如果只有造房子的梦想,而没有砖头,梦想也没法实现。——俞敏洪\par
8.奋斗这一件事是自有人类以来天天不息的。——孙中山\par
9.很难说什么是办不到的事情,因为昨天的梦想可以是今天的希望,并且还可以成为明天的现实。——罗伯特\par
10.如果我们能够为我们所承认的伟大目标去奋斗,而不是一个狂热的、自私的肉体在不断地抱怨为什么这个世界不使自己愉快的话,那么这才是一种真正的乐趣。——萧伯纳\par
11.一个人必须经过一番刻苦奋斗,才会有所成就。——安徒生\par\par
12.梦想一旦被付诸行动,就会变得神圣。——阿·安·普罗克特\par
13.凡事欲其成功,必要付出代价:奋斗。——爱默生\par
14.只有这样的人才配生活和自由,假如他每一天为之而奋斗。\par
15.不经一翻彻骨寒,怎得梅花扑鼻香。\par
16.世上没有绝望的处境,只有对处境绝望的人。\par
\subsection{相关事例}
1.贝多芬\par
大作曲家贝多芬由于贫穷没能上大学,十七岁是患了伤寒和天花病,二十六岁,不幸失去了听觉,在爱情上也屡受挫折。在这种情况下,贝多芬发誓"要扼住生命的咽喉。"在与命运的顽强搏斗中,不断奋斗,勇敢追梦,在乐曲创作事业上,他的生命之火燃烧得越来越旺盛了。逆境不但没有吓倒他,反而成了他获得强大生命力的磁场!\par
2.博格斯\par
伯格斯从小酷爱篮球,虽然身高仅有1米6,但是他经历了常人难以想象的痛苦训练。不断的奋斗和坚持他的梦想,最终成为了心所向往之人。博格斯不仅是现在NBA里最矮的球员,也是NBA表现最杰出,失误最少的后卫之一,不仅控球一流,远投神准,甚至在高个队员面前带球上篮也毫无畏惧。鼓舞了平凡人内心的意志。\par
3.江梦南\par
半岁时,江梦南检查出双耳患有极重度神经性耳聋,在父母的悉心教育下,她通过读唇语学会了听和说。靠学习唇语、看老师板书和自学,江梦南不仅高考时以615分的成绩考入吉林大学药学院,还多次获得奖学金,之后继续在吉林大学攻读完研究生学位,并且在2018年9月进入清华大学医学院攻读博士学位,致力于解决生命健康难题。永不言败的她用自己的经历,生动地诠释着有为青年的自强无畏。\par
4.张海迪\par
张海迪,小时候因患脊髓血管瘤导致高位截瘫。15岁时,她跟随父母,下放山东,自学针灸医术,为乡亲们无偿治疗。在残酷的命运挑战面前,她没有沮丧,自学了大学英语、日语、德语,并攻读了大学和硕士研究生的课程。1983年,张海迪开始从事文学创作,她以顽强的毅力克服着病痛,执着地为文学而战。她在《中国青年报》发表《是颗流星,就要把光留给人间》,名噪中华,获得两个美誉。随后,张海迪成为全国政协委员,从事创作和翻译。\par
5.屈原\par
屈原小时侯不顾长辈的反对,不论刮风下雨,天寒地冻,躲到山洞里偷读《诗经》.经过整整三年,他熟读了《诗经》305篇,从这些民歌民谣中吸收了丰富的营养,终于成为一位伟大诗人。\par
6.司马迁\par
司马迁,遵从父亲遗嘱,立志要写成一部能够"藏之名山,传之后人"的史书。就在他着手写这部史书的第七年,发生了李陵案。贰师将军李陵同匈奴一次战争中,因寡不敌众,战败投降。司马迁为李陵辩白,触怒汉武帝,被捕入狱,遭受残酷的"腐刑"。受刑之后,曾因屈辱痛苦打算自杀,可想到自己写史书的理想尚未完成。于是忍辱奋起,前后共历时18年,终于写成了《史记》。这部伟大著作共526500字。开创了我国纪传体通史的先河,史料丰富而详实,历来受人们推崇。鲁迅曾以极概括的语言高度评价《史记》:"史家之绝唱,无韵之离骚"。\par
7.诺贝尔\par
诺贝尔决心把烈性炸药改造成安全炸药。1862年夏天,他开始了对硝化甘油的研究。这是一个充满危险和牺牲的艰苦历程。死亡时刻都在陪伴着他。在一次进行炸药实验时发生了爆炸事件,实验室被炸的无影无踪,5个助手全部牺牲,连他最小的弟弟也未能幸免。但是,诺贝尔百折不挠,他把实验室搬到市郊湖中的一艘船上继续实验。经过长期的研究,他终于发现了一种非常容易引起爆炸的物质——雷酸汞,他用雷酸汞做成炸药的引爆物,成功地解决了炸药的引爆问题,这就是雷管的发明。它是诺贝尔科学道路上的一次重大突破。\par
8.儒勒·凡尔纳\par
科学幻想小说家儒勒·凡尔纳,为了写作《月球探险记》,就认真阅读了500多种图书资料。他一生之中共创作了104部科幻小说。读书笔记达二万五千本。\par
9.苏炳添\par
苏炳添在第三十二届夏季奥林匹克运动会男子百米赛场"飞人"大战中一飞冲天,成功圆梦。梦想的实现,浸透着奋斗的汗水。百米跑道上,运动员每快0.001秒,都需要付出艰辛的努力。每一次自我超越,都离不开对梦想的执着和日复一日的坚持。每次站上赛场,苏炳添拿着卷尺测量起跑器距离的细节已经广为人知。而很多人看不到的是,训练场上他一遍又一遍地蹬踏起跑器,一次又一次地回看录像。压低身体向前,起身,冲出跑道,再回到起点,蹲身,冲出跑道……每个动作都全神贯注、精益求精,每场训练都全力以赴、力求突破。成千上万次的锤炼。他告诉我们,拼搏不只在运动赛场,坚持付出、终有收获,那些超越自我、顽强拼搏的故事,必将激励我们以奋斗成就梦想,朝着更美好的生活努力进发。\par
10.中国女排\par
从"滚上一身泥,磨去几层皮,苦练技战术,立志攀高峰"的竹棚起步,中国女排在不同时期激励着几代人。习近平总书记在会见获得2019年女排世界杯冠军的中国女排队员、教练员代表时指出:"广大人民群众对中国女排的喜爱,不仅是因为你们夺得了冠军,更重要的是你们在赛场上展现了祖国至上、团结协作、奋斗拼搏、永不言败的精神面貌。女排精神代表着一个时代的精神,喊出了为中华崛起而拼搏的时代最强音。\par
\newpage
\section{规则与自由}
\subsection{名人名言}
1.没有自由的秩序和没有秩序的自由,同样具有破坏性。——西奥多·罗斯福\par
2.自由不仅为滥用权力而失去,也为滥用自由而失去。——麦奇生\par
3.如果一切都任我欲为,我会有迷失在这自由深渊之感。——伊戈尔·斯特拉温斯基\par
4.自由就是做法律所许可的一切事情的权力。——孟德斯鸠\par
5.为了享有自由,我们必须控制自己。——任尔夫\par
6.一个人只要宣称自己是自由的,就会同时感到他是受限制的。如果你敢于宣称自己是受限制的,你就会感到自己是自由的。——歌德\par
7.法典就是人民自由的圣经。——马克思\par
8.谁把法律看成是枷锁,就在开始毁灭自己——爱默生\par
9.悬衡而知平,没规而知圆。——韩非子\par
10.自由的目的是为他人创造自由。——马拉默德\par
11.法律的目的不是废除或限制自由,而是保护和扩大自由。——洛克\par
\subsection{相关事例}
1.2021年7月,印小天忽视景区规定私自带着团队站在长城城墙上跳舞,旁边还有工作人员进行录像,这段视频被游客拍摄并且发布到网络上,一下子引起了轩然大波。网友无一不批判印小天这一不文明行为。随后景区管理部门也进行了回应,指出城墙上跳舞的禁止事项,这一定性使得印小天遭受了更多的指责和批评。长城是中华文化的遗址,每个人都需要自发保护。在城墙上跳舞,不仅有坠落的风险,更是对文物的破坏和不尊重。\par
2.2021年河南水灾夺走了许多人的生命,但在救援队进行紧急的救援时,一些博主网红在救援现场进行着直播,他们不是为了播报实时情况,而是为了自己的热度。那些博主网红不顾受灾人民的安危与救援活动,顶着帮助救援的幌子偷走了救援用的皮艇,但并未去等待转移的受灾群众,还有些人更奇葩,趴在水里,佯装自己是受灾群众,甚至,当在受灾人员急需转移的时候,这群人还要确认摄像机有没有拍摄才进行救援,并且还挡住了最主要的救援通道,让真正的搜救人员进不来。他们的行为漠视了规则,只是一味地顾全了自己的自由与权利。\par
3.1764年的一-个深夜,一场大火烧毁了哈佛的图书馆,很多珍贵的古书毁于一旦,让人痛心疾首。\par
突发的火灾把一名普通学生推到了--个特殊的位置,逼迫他必须做出选择。在这之前,他违反图书馆规则,悄悄把哈佛牧捐赠的一-本书带出馆外,准备悠哉游哉地阅读完后再归还。突然之间,这本书成为珍.本。一番激烈的思想斗争之后,惴惴不安的学生终于敲开了校长办公室的门,说明理由后郑重地将书还给了学校。校长接下来的举动令人吃惊:收下书表示感谢,对学生的勇气和诚实子以褒奖,然后把他开除出校。\par
哈佛的理念是:让校规看守哈佛,比用其他东西看守哈佛更安全有效。\par
也许有人会说:这位校长太过分了,既然别人已经道了歉,归还了书,干嘛还要开除人家?这就是校规!没有了规则,还会有今天的哈佛吗?就如同江河一-样,如果没有堤岸的约束,便不成其为江河;有了堤岸,江河才能自由地奔腾。脱离了约束的自由不是真正的自由,让规则看守的自由才是真正的自由。\par
4.风筝很自由,可以翱翔于天地之间,它飞得稳稳的,不像燕子,它可以停在空中。风筝身上连着一条线,风筝线使得风筝在线轴的牵引下平稳地飞翔,不至于飞走。可是,当这根线拉得太紧的时候,风筝便会像没头苍蝇一样,乱无目的地左冲一下,右滑一下;完全没有了线的牵引,失去平衡的风筝会很快从天上掉下来。风筝的自由,建立在线的束缚和牵引上。所谓适度的自由是让事情保持在一个平衡点上,不至于太多或者太少。线放多了,风筝由于风力不够飞不起来,而线少了呢,风筝又无法在风的作用下飞上天。没有线的风筝更不行。以此来比喻我们对生活自由的态度,那就是既不要过度束缚自己,也不能过度地放纵自己,更不能认为束缚是自由的天敌,而彻底舍弃它。在生活中,面对任何事情,都需要有规则来作为适当的约束,这样才能获得真正的自由。\par
5.一次,周恩来总理去北戴河,需要看世界地图和一些书籍。工作人员给北戴河文化馆打电话,说有位领导要看世界地图和其他一些书籍。接电话的小黄回答:"我们有规定,图书不外借,要看请自己来。"周恩来便冒雨到图书馆借书。小黄一见是周总理,心里很懊悔,总理和蔼地说:"无论谁都要遵守制度。"\par
6.曾记得在一次世乒赛上,一名瑞典选手与我国选手的一场精彩比赛,那是一场极重要的赛事,关系着两位运动员能否顺利闯入决赛,所以打得特别卖力。选手的连贯对接、锐利扣杀并没有给我留下太深刻的印象,却是那位瑞典选手,渴望登上世界乒坛宝座的一个职业国家队员,在关键时刻绽放出了人性的光辉,使我看到的任何比赛在它面前都黯然失色。\par
在比赛的收尾阶段,双方依然打得难解难分,比方各不相让,在最后的关键时刻,我国选手在一次回防中,将球匆忙打向对面,如电光火石般落在桌沿外,因为动作太急骤了,在经历了短暂的寂静之后,赛场上爆发了雷鸣般的掌声。那位瑞典选手因我国选手的失误而赢得了比赛。然而,在喧闹沸腾的赛场上,大家看到了一个孤零零但执著的手势高举着,原来是瑞典选手在向裁判和观众示意:我国选手打的是擦边球,后者才是真正的胜利者。\par
大家在明了事情的真相后,既对这位瑞典选手没有赢得比赛而感到惋惜,更以热烈的掌声回报他的坦荡和率真的人性。赛后,当记者问他为什么要举起那个手势的时候,他说:"规则,让我别无选择!"\par
7.有一次,刘少奇同志去散步,走到某炮兵阵地,想进去看看。站岗的战士不让进。随行人员上前对战士说:"少奇同志想去看看阵地。"战士认真地说:"上级有规定,要有上级指示才能看。"随行人员很生气,少奇同志却没有生气。反而笑着说:"回去吧!"说着就往回走。一边走一边告诉随行人员:"回去告诉那个战士的领导,不要批评他,他做得很对。"后来部队领导知道了,要批评那个战士,少奇同志再次让工作人员转告部队领导:"这个战士认真执行规定制度,不但不应批评,还应该表扬."\par
8.魏明帝曹睿得知了蜀将马谡占领街亭,立即派骁勇善战,曾多次与蜀军交锋的曹魏名将张郃领兵抗击,张郃进军街亭,侦察到马谡舍水上山,心中大喜,立即挥兵切断水源,掐断粮道,将马谡部队围困于山上,然后纵火烧山。蜀军饥渴难忍,军心涣散,不战自乱。结果,张命令乘势进攻,蜀军大败。马谡失守街亭,战局骤变,迫使诸葛亮退回汉中。马谡违反了诸葛亮的调度,在山上扎营,是丢失街亭的主要原因,而街亭的丢失,让蜀汉军队丧失了继续进取陕西的最好时机,作为将领,马谡需要负主要责任。诸葛亮为了安抚朝野上下,不得不用马谡的人头。诸葛亮此举意在说明马谡虽然重要,但是任何人都不能把个人凌驾于国家利益之上,虽然失了街亭,但是蜀国还能争霸天下。导致整个形势急速扭转的主要原因是马谡违亮节度,不仅打乱了军事部署,而且丧失了战略要地。诸葛亮为了做到令行禁止,不得不杀马谡来起到威慑作用,这是必须的,马谡被斩,是必然的。\par
9.伟大的无产阶级革命家列宁虽然工作繁忙,但十分注意遵守公共秩序。有一次,列宁忙碌了一个上午,处理了很多日常事务,批阅了很多文件。休息的时候,他用手摸了一下头发,发觉头发实在太长了,决定抽时间去克里姆林宫理发室理发。\par
当时,这个理发室只有两个理发师,忙不过来,很多人都坐着排队,等候理发。他问哪位同志是最后一位,准备排队等候。排队理发的同志们都知道列宁日理万机,时间极其宝贵,于是争着请列宁先理发。可是列宁却微笑着对大家说:"谢谢同志们的好意。不过这样做是不对的,每个人都应该遵守公共秩序,按照先后次序理发。"他说完后,就排到最后一位同志的后面,耐心等候理发了。\par
10.那还是在延安的时候,毛泽东主席去医院看望关向应政委。两人愉快地在病房里交谈起来。护士进来说:"同志,医生吩咐,病人要安静,不能会客。"毛主席谦和地说:"对不起,小同志。"随即辞别关向应,离开了病房。\par
11.火药整形在全世界都是一个难题,无法完全用机器代替。下刀的力道,完全要靠工人自己判断,火药整形不可逆,一旦切多了,或者留下刀痕,药面精度与设计不符,发动机点火之后,火药不能按照预定走向燃烧,发动机就很可能偏离轨道,甚至爆炸。\par
0.5毫米是固体发动机药面精度允许的最大误差,而经徐立平之手雕刻出的火药药面误差不超过0.2毫米,堪称完美,这让他的师傅都望尘莫及。\par
\newpage
\section{敬业工匠精神}
\subsection{名人名言}
1.先天下之忧而忧,后天下之乐而乐。\par
2.了却君王天下事,赢得生前身后名。\par
3.捐躯赴国难,视死忽如归。\par
4.工匠精神,是指工匠对自己的产品精雕细琢,精益求精的精神理念。\par
5.工匠不一定都能成为企业家。但大多数成功企业家身上都有这种匠精神。\par
6.只做一件事容易的很,把一件事做好就需要工匠精神。\par
7.很多人认为工匠是一种机械重复的工作者,但其实,"工匠"意味深远,代表着一个时代的气质,与坚定、踏实、精益求精相连。把做的事看成有灵气的生命体。\par
8.工匠精神的价值在于精益求精,对匠心、精品的坚持和追求,专业、专注、一丝不苟且孜孜不倦。这就是工匠精神,也是追求极致的精神。其利虽微,却长久造福于世。\par
9.以精益求精之决心,以持之以恒之耐心,以爱岗敬业之忠心,以守正创新只雄心,得此四心者,可谓之工匠!\par
10.一思尚存,此志不懈。—胡居仁\par
11.神圣的事业是痛苦的,但是,也唯有这种痛苦能把深度给予我们。—张晓风《行道树》\par
12.船锚是不怕埋没自己的。当人们看不见它的时候,正是它在为人类服务的时候。—普列汉诺夫\par
13.手工时代的中国工匠相信愿力无边,不管是做佛像,还是打家具。即使只是打造一个金丝楠木柜子,可能都不是一个工匠一生就能做完的。往往是爷爷做出粗坯,父亲做完粗工,孙子再精雕细琢,穷尽三代才打造出一件精湛的柜子。陆续建造了一千六百年的莫高窟,那是多少代无名工匠,用尽了自己的体温去焐热了菩萨的慈悲\par
14.真正的科学精神,是要从正确的批评和自我批评发展出来的。真正的科学成果,是要经得起事实考验的。有了这样双重的保障,我们就可以放心大胆地去做,不会自掘妄自尊大的陷阱。——李四光\par
15.所谓匠心就是一生只做一件事,初心不改,从一而终。我们的先辈秉持带着精益求精,追求极致的工匠精神,筑就了世界的辉煌。\par
\subsection{相关事例}
1.俄国化学家门捷列夫,是自然科学基本定律化学元素周期律的发现者。门捷列夫年过七旬以后,由于积劳成疾,双目半盲,但他依然奋斗不息。他每天从清晨开始工作,一口气写作到下午5点半,6点吃"中饭",饭后又继续写到深夜。1907年1月20日清晨5时,门捷列夫因心肌梗塞坐在椅子上去世,面前的写字桌上,是一本本写完的著作。他死去时,手里还紧握着笔。\par
2.克劳德15岁时,他就嘱咐仆人,每天凌晨用这样的话唤醒自己:"克劳德先生,起来吧,伟大的事业在等待着您。"他用自己所确立的远大目标激励自己,帮助自己战胜了惰性,不断地投身于自己所创造的事业中去,终于成为杰出的空想社会主义者,在人类发展史上留下了自己探索的足迹。\par
3.卢仁峰\par
1986年,卢仁峰在某军品生产攻坚中意外发生工伤,左手4级伤残基本不能工作。重返岗位后,他定下每天练习100根焊条的底线。为了克服左手残疾带来的技术"短板",他把筷子当成焊条、把桌子当成练习试板,反复训练恢复技术能力,最终创造了熔化极氩弧焊、微束等离子弧焊、单面焊双面成型等操作技能,《短段逆向带压操作法》《特种车辆焊接变形控制》等多项成果,"HT火花塞异种钢焊接技术"等国家专利。他牵头完成152项技术难题攻关,提出改进工艺建议200余项,一批关键技术瓶颈的突破为实现强军目标贡献了智慧和力量。\par
4.刘丽\par
刘丽始终把"我为祖国献石油,保障国家能源安全"作为己任,坚守在生产一线,苦练本领。她专注于解决生产难题,研发各类成果200余项,其中获国家及省部级奖项33项、国家专利及知识产权软著41项。她研制的"上下可调式盘根盒",使操作时间缩短四分之三,填料使用寿命延长6倍,在60000多口油井应用,年节约维修工时10万小时、节电2.4亿多度。她研发的"螺杆泵井新型封井器装置"等一批成果填补了国际国内技术空白,累计多产油60000多吨。\par
5.火箭"心脏"焊接人高凤林\par
今年53岁的高凤林,是中国航天科技集团公司第一研究院211厂发动机车间班组长,35年来,他几乎都在做着同样一件事,即为火箭焊"心脏"——发动机喷管焊接。130多枚长征系列运载火箭在他焊接的发动机的推动下顺利进入太空,其中就有送嫦娥卫星去月球的长征三号甲系列火箭。0.08毫米,是高凤林焊接生涯里挑战过的最薄记录。有的实验,需要在高温下持续操作,焊件表面温度达几百摄氏度,高凤林却咬牙坚持,双手被烤得鼓起一串串水泡。因为技艺高超,曾有人开出"高薪加两套北京住房"的诱人条件聘请他,高凤林却说,我们的成果打入太空,这样的民族认可的满足感用金钱买不到。他用35年的坚守,诠释了一个航天匠人对理想信念的执着追求。
6."蛟龙号"上的"两丝"钳工顾秋亮\par
顾秋亮,1972年起在中国船舶重工集团公司第702研究所工作,在钳工安装及科研试验工作方面已经工作了四十多年。"蛟龙号"是中国首个大深度载人潜水器,有十几万个零部件,组装起来最大的难度就是密封性,精密度要求达到了"丝"级。而在中国载人潜水器的组装中,能实现这个精密度的只有钳工顾秋亮,也因为有着这样的绝活儿,顾秋亮被人称为"顾两丝"。43年来,他埋头苦干、踏实钻研、挑战极限,追求一辈子的信任,这种信念,让他赢得潜航员托付生命的信任,也见证了中国从海洋大国向海洋强国的迈进。2009年至2012年,顾秋亮作为蛟龙号海上试验技术保障骨干,全程参与了蛟龙号载人潜水器1000米、3000米、5000米和7000米四个阶段的海上试验。参加海上试验时,顾秋亮已是五十多岁,但他克服了严重的晕船反应和海上艰苦的工作生活条件等诸多困难,安排好家中生病的妻子,义无反顾地投入到每年近100天的海试中。他带领装配保障组不仅完成了蛟龙号的日常维护保养,还和科技人员一道攻关,解决了海上试验中遇到的技术难题,如压载铁的安装、水下灯光的调整、布放回收接口的设置等,并将自己的技术和心得体会毫无保留地传授给国家深海基地的技术人员,为海试的顺利进行和蛟龙号投入正规化的业务运行立下了汗马功劳。\par
顾秋亮说:"在海上工作生活确实很苦很累,但我感到很兴奋、很自豪。不管是晚上加班到半夜还是早上五点半起床保养潜器,不管日晒还是雨淋,我感到很光荣,能为海试出一份力,我很骄傲,因为在祖国的深潜记录中有我的汗水,光荣!"\par
怀揣崇高的使命感和荣誉感,他又肩起了新的挑战——组装4500米载人潜水器。已近花甲的顾秋亮仍坚守在科研生产第一线,为载人深潜事业不断书写我国深蓝乃至世界深蓝的奇迹默默奉献……\par
7.张黎明作为一名基层的修复电路工人,从抢修到创新,从管理到月服务,都认真负责,深受群众喜爱。面对暴风雨,他坚守对工作的职责,仍然解决居民电路问题;将自己的电话留给居民,帮助老人,为片区居民一心一意服务;时刻牢记自己是共产党员,成为居民委员会的的模范带头人。张黎明在平凡的岗位书写不平凡的故事,将日复一日的工作做的崇高,成为点亮万家的蓝领工匠。让我们看到他对工作责任的肩负,让我们看到小人物身上所具有的大国工匠精神。\par
8.七旬教师呼秀珍被称为"永不退休的雷锋""捧着一颗心来,不带半半颗草去",她将毕生热情投入教育事业。从埋首三尺讲桌到奔赴四处家访,长路迢迢,她坚持走好每一步,只为拉近师生心的距离。"认认真真教课,全心全意育人"是她对人生信念的笃定践行。每个普通个体以乎都是平凡而渺小的,却也是维持社\par
会动脉运转不可缺少的血液。\par
9.19岁就成为纺织工人的李兰女,对于工作,她勤勤恳恳,不断学习,她的青春,没有被"赚钱养活自己就行"的苟且所占据,相反,她悉心钻研技术与实践,不仅靠自己的努力得到了大专院校毕业证书,更成为了纺织业中的革新达人。\par
\newpage
\section{责任与担当}
\subsection{名人名言}
1、我们为祖国服务,也不能都采用同一方式,每个人应该按照资禀,各尽所能。——歌德\par
2、社会犹如一条船,每个人都要有掌舵的准备。——易卜生\par
3、责任就是对自己要求去做的事情有一种爱。——歌德\par
4、有许多东西,只要我们对它们陷入盲⽬性,缺乏⾃觉性,就可能成为我们的包袱,成为我们的负担。——⽑泽东\par
5、作为确定的人,现实的人,你就有规定、就有使命、就有任务,至于你是否意识到这⼀点,那是无所谓的。——马克思\par
6、天下兴亡,匹夫有责。——顾炎武\par
7、真正进步的人,决不以"孤独"、"进步"为⼰⾜,必须负起责任,使⼤家都进步,至少使周围的人都进步。——邹韬奋\par
8.人生须知负责任的苦处,才能知道有尽责的乐趣。——梁启超\par
9.有良知的人有责任心和事业心。——苏霍姆林斯基\par
10.责任心不是蓝天上的白云,潇洒,飘逸,片刻消失,而是万物生存必不可少的甘露。\par
11.高尚、伟大的代价就是责任。——丘吉尔\par
12.一个人若是没有热情,他将一事无成,而热情的基点正是责任心。——列夫.托尔斯泰\par
13.每一个人都应该有这样的信心:人所能负的责任,我必能负;人所不能负的责任,我亦能负。如此,你才能磨炼自己,求得更高的知识而进入更高的境界。——林肯\par
14.为人民服务,担当起该担当起的责任。——习近平\par
\subsection{相关事例}
1.在华盛顿年幼时,有一天父亲外出,他用小斧头砍倒了院子里的一颗樱桃树。那是父亲最喜爱的樱桃树,父亲回来后见心爱的树被砍,非常气愤,扬言要给那个砍树的一顿教训。而华盛顿在盛怒的父亲面前毫不避地承认了自己的错误。于是父亲被小华盛顿感动了,并原谅了华盛顿,称他的诚实勇敢比所有樱桃树都宝贵得多。\par
2.一个少女到东京帝国酒店做服务员,这是她涉世之初的第一份工作。但她万万没有想到上司安排她洗厕所!上司对她工作质量的要求特别高:必须把马桶抹洗得光洁如新。\par
一位先辈看到她的犹豫态度,不声不响地为她做了示范,当他把马桶洗得光洁如新时,他竟然从中舀了一碗水喝了下去。先辈对工作的态度,使她明白了什么是工作,什么是责任心,从此她漂亮地迈出了职业生涯的第一步,并踏上了成功之路。自然,她所清洗的厕所,一向光洁如新。几十年一瞬而过,如今她已是日本的邮政大臣。她的名字叫野田圣子。\par
3.黄旭华⼀⼀中国核潜艇之父。⼯程复杂,没有计算机,他就和同事⽤算盘和计算尺演算成千上万个数据。带领团队研制出我国第⼀艘核潜艇。使中国成为世界上第五个拥有核潜艇的国家。核潜艇按设计极限在南海作深潜试验。黄旭华亲⾃下潜300⽶,是世界上核潜艇总设计师亲⾃下水做深潜试验的第⼀人。\par
4.鲁迅改稿:鲁迅先生提倡写文章"写完后至少看两遍,竭力将可有可无的字、句、段删去,毫不可惜"他自己身体力行着这一主张,直到生命的最后一刻。在他生命的最后两天中所写的《因太炎先生而想起的二三事》一文的修改上,清楚地表现了这一点。当时,他"已经没有力气了"但他仍坚持修改,在这篇最终未能完成的仅有2600多字的短短文稿中,修改的痕迹竟达53处之多。\par
5.陈金水献身祖国气象事业:陈金水从气象学院毕业后,离开山清水秀的浙江只身来到青藏高原。他在世界屋脊建立起世界上最高的气象站。在卧室里悬挂着"祖国的气象事业高于一切"的横幅,以表明自己的心迹。他是这样说也是这样做的。在青藏高原一干就是30年。青藏高原生活环境极为艰苦,终年积雪,万里无人。由于低压高寒,他吃不上煮熟的饭,吃不到新鲜蔬菜。由于缺氧,落下了心血管疾病。他为青藏高原的气象事业,做出了开创性的贡献。\par
6.林则徐,于列强横行之时,万里销烟,雄壮虎门,壮我国人。是他,冷对昏君,怒对贪官,担当起"开眼看世界"的责任。是他,含恨被贬,忠而被忘,仍心念强国图存的责任!"苟利国家生死以,岂因祸福避趋之",他拥有的是一颗英武的民族魂。\par
7.伟大的科学家钱学森年轻时留学美国,学有所成后,不顾美国政府的反对,执意回国效力。他放弃了优越的生活环境,放弃了高额薪水,双手重新抓住的却是中国导弹事业,扶起的亦是中华民族不屈的灵魂,担当起名族复兴大任。\par
8.伟大的科学家钱学森年轻时留学美国,学有所成后,不顾美国政府的反对,执意回国效力。他放弃了优越的生活环境,放弃了高额薪水,双手重新抓住的却是中国.导弹事业,扶起的亦是中华民族不屈的灵魂。\par
9.雷锋,他是一名平平常常的人民子弟兵,他没有显著的家庭背景,没有过人的头脑与才华。但他却成了中华人民的榜样,而这样的成就,就与他一生的行为举动及言语有关,他从来没有与人民发生嘴角的摩擦。他常常把他的工资捐给一些孤儿和贫困的人,而他吃穿都很简单,正因他处处为人民着想,用于担当,使他成了中华人民永远的骄傲。\par
10.2003年非典肆虐,67岁的他说"把最重的病人送到我这来。"而今,武汉疫情肆虐,84岁的他一边告诉公众"尽量不要去武汉",一边自己却登上了前往武汉的高铁。面对未知的艰难,他勇担大任,一往无前,既有医者仁心的专业技术,也有战士迎难而上的拼搏狠劲。但是在他看似无坚不摧的外衣之下,却深藏一颗柔情似水的仁人之心、一腔热血爱国的赤诚之情。\par
\newpage
\section{和谐、绿色发展}
\subsection{名人名言}
1.恶德——不和、战争、悲惨;美德——和平、幸福、和谐。——雪莱\par
2.美在和谐。——赫拉克利特3.只有服从大自然,才能战胜大自然。——雨果\par
3.大地给予所有的人是物质的精华,而最后,它从人们那里得到的却是这些物质的垃圾。——惠特曼\par
4.大自然是善良的慈母,同时也是冷酷的屠夫。——雨果\par
5.人们常常将自己周围的环境当作一种免费的商品,任意地糟蹋而不知加以珍惜。——甘哈曼\par
6.幸福永远存在于人类不安的追求中,而不存在于和谐与稳定之中。——鲁迅\par
7.亲善产生辛福,文明带来和谐,友谊是一种和谐的平等。——雨果\par
8居家有二语,曰:惟怒则平情,惟俭则足用。——洪应明\par
9.和谐是众多因素的统⼀,不协调因素的协调。——毕达哥拉斯\par
11.和谐是爱与恨结合起来的庄严的配偶。——罗曼·罗兰\par
12.美的真谛应该是和谐。这种和谐体现在人身上,就造就了人的美;表现在物上,就造就了物的美;融汇在环境中,就造就了环境的美。——冰心\par
13.友谊是一种和谐的平等。——毕达哥拉斯\par
14.非但不能强制自然,还要服从自然。——埃斯库曼斯\par
\subsection{相关事例}
1.清朝康熙年间有个大学士名叫张英.一天张英收到家信,说家人为了争三尺宽的宅基地,与邻居发生纠纷,要他用职权疏通关系,打赢这场官司.张英阅信后坦然一笑,挥笔写了一封信,并附诗一首:千里修书只为墙,让他三尺有何妨?万里长城今犹在,不见当年秦始皇.家人接信后,让出三尺宅基地.邻居见了,也相让三尺宅基地.结果成了六尺巷,这个化干戈为玉帛的故事流传至今,这则故事告诉我们要有坦荡的胸怀,人与人之间要保持一种和谐的人际关系。\par
2.战国时赵国舍人蔺相如奉命出使秦国,不辱使命,完壁归赵;又陪同赵王赴秦王设下的渑池会,使赵王免受暗算。为奖励蔺相如的汗马之功,赵王封蔺相如为丞相。老将廉颇居功自傲,对此不服,而屡次故意挑衅,蔺相如以国家大事为重,始终忍让。后廉颇终于醒悟,向蔺相如负荆请罪。将相和好,共同辅国,国家无恙。\par
3.青蛙是害虫的天敌,稻田里祸害稻禾们的害虫,都是青蛙的美食,所以青蛙被誉为"农田守护神"。可是,现在有很多人将青蛙当做美味野味,大量捕杀青蛙,将这些"农田的守护神"摆上餐桌,做成"田鸡粥""田鸡煲"等美味食物,一饱口福。但是没有青蛙的守护,农田的稻禾们就会因为害虫的侵食而失收,人们赖以生存的粮食作物,就会因为减产而造成市场价格价格飙升,从而造成供不应求,而引致可怕的饥荒,进而导致社会动荡,民不聊生。\par
4.孔融,字文举,东汉时期山东曲阜人,是孔子的第二十世孙,高祖父孔尚当过钜鹿太守,父亲是泰山都尉孔宙。孔融别传记载:孔融四岁的时候,和哥哥吃梨,总是拿小的吃。有人问他为什么这么做。他回答说:"小孩子食量小,按道理应该拿小的。"\par
5.丁绍光是著名的旅美画家,他的画以和谐为主题,成就巨大。他觉得从哲学、文艺发展以及人类历史的角度看,和谐是重要的,这也是现在的世界所欠缺的。丁绍光也曾画过反映内心冲突的画,但那是在年轻时候,到国外后,他觉得世界缺少和谐,所以从此以和谐为主题作画。\par
6.新县地处豫南大别山腹地,是国家扶贫开发工作重点县和大别山集中连片特困地区扶贫攻坚重点县。近年来,新县大力实施生态立县战略,围绕"山水红城、健康新县"的发展定位,牢树"红色引领、绿色发展"理念,倡树"视山如父、视水如母、视林如子"生态意识,念好"山字经"、唱好"林中戏"、打好"生态牌"、走好"特色路",持续推动"生态资本"转化为"富民资本",探索打通"绿水青山就是金山银山"转化通道。获得国家卫生县城"五连冠"、国家园林县城"四连冠"、全国文明城市"三连冠"和国家生态文明建设示范县等10多项国家级荣誉。\par
7.老子说,"人法地,地法天,天法道,道法自然",人法乎地、地法乎天、天法乎道,而道法乎自然。人、地、天、道、自然,这是人类认知这个世界的演进路径,随着这一演进路径向前走的,是人类对这个世界的认识范围的不断扩大。\par
\newpage
\section{爱国}
\subsection{名人名言}
1.我爱中国固因他是我的祖国,而尤因他是有那种可敬爱的文化的国家。——闻一多\par
2.爱国主义就是千百年来巩固起来的对自己的祖国的一种最深厚的感情。——列宁\par
3.爱国心再和对敌人的仇恨用乘法乘起来——只有这样的爱国心才能导向胜利。——奥斯特洛夫斯基\par
4.每一个伟大人物的历史意义,是以他对祖国的功勋来衡量的,他的人品是以他的爱国行为来衡量的。——车尔尼雪夫斯基\par
5.黄金诚然是宝贵的,但是生气勃勃、勇敢的爱国者却比黄金更为宝贵——林肯\par
6.与其忍辱生,毋宁报国死——何香凝\par
7.祖国的命运就是自己的命运!——常香玉\par
8.为了抉择真理,为了国家民族,我要回过去!——华罗庚\par
9.人类最高的道德标准是什么?那就是爱国心。一一拿破仑\par
10.祖国如有难,汝应作前锋。一一陈毅\par
11.爱祖国高于一切。一一肖邦\par
12.我们爱我们的民族,这是我们自信心的源泉。——周恩来\par
13.人民不仅有权爱国,而且爱国是个义务,是一种光荣。——徐特立\par
14.唯有民魂是值得宝贵的,唯有他发扬起来,中国才有真进步。——鲁迅\par
15.我荣幸地从中国民族一员的资格,而成为世界公民。我是中国人民的儿子,我深情地爱着我的祖国和人民。——邓小平\par
16.锦城虽乐,不如回故乡;乐园虽好,非久留之地。归去来兮。——华罗庚\par\par
\subsection{相关事例}\par
1.杨靖宇\par
民族抗日英雄杨靖宇曾担任"南满抗日联军"司令,从1934年一直到1940年沙场献身为止。在艰苦征战的六年中,他身先士卒地在白山黑水、林海雪原里打击日寇。面对敌人的重兵围剿,杨靖宇率部顽强战斗,使敌人坐卧不安,惶惶不可终日。日寇对他又怕又恨,调集重兵围困。有人劝杨靖宇投降,他斩钉截铁地说:"不,我有我的信念。"最后,弹尽粮绝,杨靖宇在打完最后一颗子弹后壮烈牺牲。敌人残忍地用刺刀剖开他的肚子,杨靖宇肚里没有一粒米,有的只是树皮、草根和棉絮。\par
2.詹天佑为国不计名与利:\par
近代科学先驱、著名工程师詹天佑,在国内一无资本、二无技术、三无人才的艰难局面面前,满怀爱国热情,受命修建京张铁路。他以忘我的吃苦精神,走遍了北京至张家口之间的山岭,只用了500万元、4年时间就修成了外国人计划需资900万元、需时7年才能修完的京张铁路。\par
前来参观的外国专家无不震惊和赞叹。当时,美国有所大学为表彰詹天佑的成就,决定授予他工科博士学位,并请他参加仪式。可是,詹天佑正担负着另一条铁路的设计任务,因而毅然谢绝了邀请。他这种为国家不为个人功名的精神,赢得了国内外的称赞。\par
3.张伯苓\par
南开中学的创办者张伯苓16岁时以优异的成绩考入北洋水师学堂,学习驾驶。毕业后,他参加了"甲午海战",但军舰一出海就被击沉,这对他触动很大。1899年英国强租我国威海卫军港,张伯苓亲眼看见,第一天在港口升起的清朝国旗第二天就降下来了。强烈的爱国心促使他毅然退出海军,回到天津筹办学校。他四处奔走,筹集资金,终于在1907年办起了南开学校。张伯苓一生全力办教学为国家培养了大批的人才。\par
4.华罗庚\par
著名数学家华罗庚早年在美国很受学术界器重。有人想和他签订合同,把他留在美国,给予优厚的待遇,但当他得知新中国成立的消息后,立即决定回国。途经香港时,他发表了一封给留美学生的公开信,满怀热情地呼吁他们:"为了国家民族,我们应当回去!"\par
5.刘德华\par
刘德华一行来到日本,举办了小型歌友会。歌友会一开始,歌迷欢呼。但是刘德华拒绝用日语向大家问好。接着又拒绝了主办方安排的日语歌曲。并把所有曲目都改为普通话。原定的粤语歌曲也全部取消。刘德华用普通话一字一句说道:"本来我是不想来日本的,但是因为合约在身,不得不来日本。但是你(指记者)不要以为是一纸合约把我牵住的,如果我不来,没人可以把我怎么样。我只是觉得这样对我歌迷不公平,因为歌迷是无辜的。我不想做的事,谁也无法逼我做,而且,你逼我,我也不会做"。刘德华接着说:"引用一句话,艺术是没有国界的,但是艺术家是有国界的。我想说,音乐是没有国界的,但是音乐家是有国界的。"他对着主办方说"以后介绍我时,不要说我是香港歌手,因为我首先是一个中国人"。然后,他当着几百名日本歌迷的面,演唱了一首《中国人》\par
6.张自忠是为国捐躯的将军,是"抗战军人之魂"。\par
张自忠经常教育部下:军人只有以必死的决心去战胜敌人,才能对得起国家和自己的良心。1940年5月,民党军三十三集团军总司令张自忠率军在湖北襄樊一带抗战。大洪山一战,他们消灭了1000多名日寇。日军疯狂报复,派重兵包围过来。张自忠几次率军反击,没有成功。张自忠左肩受了伤,他说:"我是不打败仗的,败只有死,我不能对不起部下。只有誓死不退,才能抗敌保国。"说完,他又顽强地站起来,向敌人扑过去,敌人的子弹又射中他的腹部和头部。张自忠为国尽忠了,他是在抗战中牺牲的中国军人中职务最高的一个。\par
7.钱学森\par
钱学森在美国度过了20年,在航空科学上取得了卓越的成就,成为有名的火箭专家,为美国的军事科学做出了贡献。1949年,他得知新中国成立了,非常兴奋,决定回国参加建设。美方就干方百计地阻挠。美国海军次长还恶狠狠地说:"我宁肯把他枪毙了,也不让他离开美国。他知道的太多了,一个人可顶五个师的兵力!"于是,把他捕关押,钱学森没有屈服,回国的决心更大了。他在家里放好三只小箱子,准备随时启程。后来在中国政府的过问下,被美方扣留了5年的钱学森,终于在1955年搭乘轮船回国了。他来到天安门广场,兴奋地说:"我相信我一定能回来,现在终于回来了!"钱学森回国后,为我国导弹和航天事业做出了巨大贡献,是最有声望的科学家之一。\par
8.袁隆平\par
袁隆平是中国工程院院士,著名杂交水稻专家,国家科学技术奖获得者,中国第一个国家特等发明奖获得者。在国际上11次捧回大奖。获得的"世界粮食奖"更是农业领域国际上的最高荣誉。\par
袁隆平精神:朴实厚道,务实求,能甘于寂寞、吃苦耐劳、不怕失败、不屈不挠。\par
袁隆平是我国杂交水稻研究创始人,被誉为"杂交水稻之父"、"当代神农"、"米神"等。\par
9.杨伟\par
歼20,中航工业成都飞机设计研究所研制的一款具备高隐身性、高态势感知、高机动性等能力的隐形第五代制空战斗机,他的总设计师杨伟。15岁进入西工大,学习非常努力,待到毕业之后,杨伟和许多人一样也想着出国去深造,毕竟优秀的人才都会选择去国外进修下,然后继续回到祖国工作。\par
可是这个时候"歼10之父"宋文骢却劝告杨伟别走了,且让他留在国内学习和发展,就这样杨伟成为了航空工业成都所九室设计员。\par
自此以后杨伟将自己全部精力都注射到了中国航空事业中,谁都没想到的是在之后的科研工作中,杨伟确实是为祖国的航天事业立下了巨大的功劳,他用了1年的时间就攻克了电传飞控系统技术难关。2001年开始起,杨伟出任了航空工业成都所总设计师,在他的领导下,2011年1月,歼20年首飞成功。\par
10.茅以升\par
在我国老一辈科学家中,有许多人都是留学国外又回国服务的。著名桥梁专家茅以升在1916年20岁时,到美国留学,成为康奈尔大学桥梁专业的研完生,很快以优异的成绩获得硕士学位。为了获得实践的机会,他晚上上课,攻读博士学位,白天到一家桥染公司实习,亲手绘图、切削钢件、打铆钉、油漆,终于成了一个既懂理论又有技术的人才。美国人很佩服他,一份份聘书从各地奇来,请他担任工程师。但是,茅以升没有接受聘请,而是决定回国了。美国有些人劝他:"科学是没有祖国的,是超越国界的。科学家的贡献是屬于全人类的。中国条件差:你留在美国贡献会重大。〞茅以升回答:"科学是没有祖国的,但是科学家是有祖国的。我是一个中国人,我的祖国更需要我。我要回去为祖国服务!1919年,茅以升带着一身本领回到国内,开始了为国造桥的事业。现在浙江省钱塘江上那座雄伟壮观的大桥,就是茅以升设计并支持建造的。\par
11.肖邦\par
肖邦⼀⽣不离钢琴,所有创作⼏乎都是钢琴曲,被称为"浪漫主义的钢琴诗⼈"。他在国外经常为同胞募捐演出,为贵族演出。1837年严辞拒绝沙俄授予他的"俄国皇帝陛下⾸席钢琴家"的职位。舒曼称他的⾳乐像"藏在花丛中的⼀尊⼤炮",向全世界宣告:"波兰不会亡"。肖邦晚年⽣活⾮常孤寂,痛苦地⾃称是"远离母亲的波兰孤⼉"。他临终嘱附姐姐路德维卡把⾃⼰的⼼脏运回祖国。\par
12.陈天华\par
⾰命家陈天华,在⽇本留学时,听到沙俄军队侵占满洲,腐败⽆能的清政府⼜要同沙俄私订丧权辱国条约的消息后,他悲愤欲绝,⽴即在留学⽣中召开拒俄⼤会,组织拒俄义勇军,准备回国参战。回到宿舍后,咬破⾃⼰⼿指,以⾎指书写救国⾎书,在⾎书⾥陈述亡国的悲惨,当亡国奴的⾟酸,⿎舞同胞起来战⽃……他⼀连写了⼏⼗张,终因流⾎过多⽽晕倒,可嘴⾥还在不停地咸:"救国!救国!"。别⼈把他救醒后,他坚持把⾎书⼀份⼀份装⼊信封,从万⾥迢迢的⽇本寄回国内,读到的⼈⽆不感动。\par
\newpage
\section{榜样的力量}
\subsection{名人名言}
1.典范的力气是无量的。–科·达勒维耶\par
2.典范具有杰出的感染力。–塞·约翰逊\par
3.以身作则胜于口头训诲。——英国\par
4.教诲是条漫长的道路,榜样是条捷径。——塞内加\par
5.以身教者从,以言教者讼。——后汉书\par
6.青年的思想愈被范例的力量所激励,就愈会发出强烈的光辉。——法捷耶夫\par
7.命令只能指挥人,榜样却能吸引人。——威·亚历山大\par
8.人不率,顺不从;身不先,则不信。——宋史\par\par
9.好榜样就像把许多人召集到教堂去的钟声一样。——丹麦\par
10.一个榜样胜过书上二十条教诲。——罗·阿谢姆\par
11.既然真理和坚贞均告徒劳,既然爱情、痛苦和理智的力量都不能将其说服那么就让榜样作为儆戒吧!——乔·格兰维尔\par
12.一个人所能做的就是做出好榜样,要有勇气在风言风语的社会中坚定地高举伦理的信念。——爱因斯坦\par
13.教,上所施,下所效也。——许慎\par
14.榜样的力量是无穷的。——科达勒维耶\par
\subsection{相关事例}
1.班超曾经弃笔感叹:"大丈夫无它志略,犹当效傅介子、张骞立功异域,以取封侯,安能久事笔研间乎?"后来班超以三十六人纵横西域,在艰难的环境下当机立断、百战不殆,写下传奇人生,更创造了中国历史名将之西域神话,扬我中华国威,牵制匈奴保汉帝国太平,制止匈奴在西域的恃强凌弱,更维护西域诸国和平往来。\par
2.张海迪五岁患脊髓病,胸部以下全部瘫痪。残酷的命运面前,她没有消沉。她尽管没有机会走入校门,却发奋学习,攻读了大学与研究生的课程后致力于文学创作。为了更好的奉献社会,她甚至自学了十几类医学类论著,为人民群众无偿医治达1万多人次。是什么让一个残疾女孩为之如此努力与奉献?她说,这是因为她以保尔、雷锋为楷模,始终坚守着"活着就要做一个对社会有利的人"的信念,奉献着自己的光与热。\par
3.3月3日晚,湖南郴州女孩江梦南当选2021感动中国年度人物。她是吉林大学药学院2011级本科生、2015级硕士研究生,一个右耳失聪、交流不便的湖南女孩,为何要选择遥远的北方大学?江梦南说,一是偶像张海迪也毕业于吉林大学,她说"要跟随海迪阿姨的脚步",同时,江梦南还一直有一个"英雄梦",她要治病救人,祛除病痛。\par
4.巴雷尼小时候因病成了残疾,母亲的心就像刀绞一样,但她还是强忍住自己的悲痛。她想,孩子现在最需要的是鼓励和帮助,而不是妈妈的眼泪。母亲来到巴雷尼的病床前,拉着他的手说:"孩子,妈妈相信你是个有志气的人,希望你能用自己的双腿,在人生的道路上勇敢地走下去!好巴雷尼,你能够答应妈妈吗?"母亲的话,像铁锤一样撞击着巴雷尼的心扉,他"哇"地一声,扑到母亲怀里大哭起来。\par
从那以后,妈妈只要一有空,就给巴雷尼练习走路,做体操,常常累得满头大汗。有一次妈妈得了重感冒,她想,做母亲的不仅要言传,还要身教。尽管发着高烧,她还是下床按计划帮助巴雷尼练习走路。\par
母亲的榜样作用,更是深深教育了巴雷尼,他终于经受住了命运给他的严酷打击。他刻苦学习,学习成绩一直在班上名列前茅。最后,以优异的成绩考进了维也纳大学医学院。最后,终于登上了诺贝尔生理学和医学奖的领奖台。\par
5.爱因斯坦父亲说,"我和咱们的邻居杰克大叔清扫南边工厂的一个大烟囱。那烟囱只有踩着里边的钢筋踏梯才能上去。你杰克大叔在前面,我在后面。我们抓着扶手,一阶一阶地终于爬上去了。\par
下来时,你杰克大叔依旧走在前面,我还是跟在他的后面。后来,钻出烟囱,我发现一个奇怪的事情:你杰克大叔的后背、脸上全都被烟囱里的烟灰蹭黑了,而我身上竟连一点烟灰也没有。"\par
爱因斯坦的父亲继续微笑着说:"我看见你杰克大叔的模样,心想我肯定和他一样,脸脏得像个小丑,于是我就到附近的小河里去洗了又洗。而你杰克大叔呢,他看见我钻出烟囱时干干净净的,就以为他也和我一样干净呢,于是就只草草洗了洗手就大模大样上街了。结果,街上的人都笑痛了肚子,还以为你杰克大叔是个疯子呢。"\par
爱因斯坦听罢,忍不住和父亲一起大笑起来。父亲笑完了,郑重地对他说,"其实,别人谁也不能做你的镜子,只有自己才是自己的镜子。拿别人做镜子,白痴或许会把自己照成天才的。"\par
爱因斯坦听了,顿时满脸愧色。爱因斯坦从此离开了那群顽皮的孩子们。他时时用自己做镜子来审视和映照自己,终于映照出生命中的熠熠光辉。\par
6.唐朝著名学者陆羽,从小是个孤儿,被智积禅师抚养长大。陆羽虽身在庙中,却不愿终日诵经念佛,而是喜欢吟读诗书。陆羽执意下山求学,遭到了禅师的反对。禅师为了给陆羽出难题,同时也是为了更好地教育他,便叫他学习冲茶。\par
在钻研茶艺的过程中,陆羽碰到了一位好心的老婆婆,不仅学会了复杂的冲茶的技巧,更学会了不少读书和做人的道理。当陆羽最终将一杯热气腾腾的苦丁茶端到禅师面前时,禅师终于答应了他下山读书的要求。后来,陆羽撰写了广为流传的《茶经》,把祖国的茶艺文化发扬光大。\par
7.卡莉·菲奥里纳:从母亲那里受益匪浅惠普前任女掌门卡莉。菲奥里纳曾是男性主导的硅谷中最亮丽的一道风景。精明强干、坚忍不拔的卡莉曾两度荣登财富"最有权威的女企业家"榜首,吸引了全世界的目光。卡莉从小就受母亲影响,从母亲那里学到了坚强、博学和热爱生活,并受益一生。卡莉出生于美国得州一个带有欧洲血统的家庭。父亲是联邦法院的法官,母亲则是一位艺术家。在童年的卡莉心中,母亲一直是她最崇敬的人。母亲热爱生活,教卡莉做人的道理,使卡莉的潜能得到最大的发挥。卡莉童年时代随父母游历了不少国家,不仅开拓了眼界,更培养了思考问题的广度和深度,这对她成为一个有勇气、有魄力、自信并热爱生活的人也不无影响。\par
8.脱口秀女王"奥普拉。\par
虽然现在她备受大家的喜爱和尊重,但小时候的她,却有着一段很痛的童年。因为父母分居,奥普拉从小就被丢给祖母抚养,没有爸妈的关心和陪伴。上了学,同学也因为她是黑人而嘲笑、捉弄她。但命运的不公,没有打败奥普拉,反倒让她变得更加强大。\par
9.王羲之13岁那年,偶然发现他父亲藏有一本《说笔》的书法书,便偷来阅读。他父亲担心他年幼不能保密家传,答应待他长大之后再传授。没料到,王羲之竟跪下请求父亲允许他现在阅读,他父亲很受感动,终于答应了他的要求。\par
王羲之练习书法很刻苦,甚至连吃饭、走路都不放过,真是到了无时无刻不在练习的地步。没有纸笔,他就在身上划写,久而久之,衣服都被划破了。有时练习书法达到忘情的程度。一次,他练字竟忘了吃饭,家人把饭送到书房,他竟不加思索地用摸摸蘸着墨吃起来,还觉得很有味。当家人发现时,已是满嘴墨黑了。\par
王羲之常临池书写,就池洗砚,时间长了,池水尽墨,人称"墨池"。现在绍兴兰亭、浙江永嘉西谷山、庐山归宗寺等地都有被称为"墨池"的名胜。\par
10.张定宇,57岁,武汉市金银潭医院院长,被授予人民英雄国家荣誉称号。武汉市金银潭医院是最早接诊新冠患者的定点医院,收治病人全部为重症和危重症患者,是抗击疫情的最前沿。身为院长的张定宇日夜坚守,果断决策,处理得当,带领全体医护人员,为抗击疫情做出重要贡献。张定宇自己还是一位病人,2018年10月他被确诊为患有渐冻症,在新冠袭击武汉时,张定宇隐瞒了病情,也无法照顾已感染新冠的妻子,一直坚守在抗疫一线。\par
11.陈陆。36岁,安徽省庐江县消防救援大队政治教导员。2020年夏天,安徽庐江县遭受百年一遇洪灾,7月22日,庐江县石大圩漫堤决口,约6500人被洪水围困,情况危急。当天,陈陆带领庐江县消防救援大队辗转5个乡镇,连续奋战,成功转移群众2665人。在营救过程中,决口突然扩大,救援队员所乘橡皮艇被卷入激流漩涡侧翻,陈陆英勇牺牲。说,他们是我们心中的英雄,我们都会崇敬他们。每当危难时刻总有英雄挺身而出,这是中华民族伟大精神的象征。\par
\end{document}